% Зачем: Предоставляет проприетарный Times New Roman.
% ОБНОВЛЕНИЕ: лучше использовать scalable-cyrfonts-tex: меньше проблем с установкой
% Из руководства к PSCyr: "Во избежание проблем пакет PSCyr должен загружаться перед пакета-ми inputenc и babel".
% Примечание: Требует шаманства при установке, инструкция http://plumbum-blog.blogspot.com/2010/06/miktex-28-pscyr-04d.html
% http://blog.harrix.org/?p=444
% \usepackage{pscyr}

% % Зачем: Выбор внутренней TeX кодировки.
% \usepackage[T2A]{fontenc}

\usepackage{fontspec}
\usepackage{xunicode}
\usepackage{xltxtra}


\usepackage{polyglossia}                        % Поддержка многоязычности (fontspec подгружается автоматически)
    \setmainlanguage[babelshorthands=true]{russian} % Язык по-умолчанию русский с поддержкой приятных команд пакета babel
    \setotherlanguage{english}                      % Дополнительный язык = английский (в американской вариации по-умолчанию)
    \setmonofont{Courier New} % моноширинный шрифт
    \newfontfamily\cyrillicfonttt{Courier New}[Script = Cyrillic] % моноширинный шрифт для кириллицы
    \defaultfontfeatures{Ligatures=TeX} % стандартные лигатуры TeX, замены нескольких дефисов на тире и т. п. Настройки моноширинного шрифта должны идти до этой строки, чтобы при врезках кода программ в коде не применялись лигатуры и замены дефисов
    \setmainfont{Arial} % Основной шрифт
    \newfontfamily\cyrillicfont{Times New Roman}[Script = Cyrillic] % Основной шрифт для кириллицы
    \setsansfont{Times New Roman} % Шрифт без засечек
    \newfontfamily\cyrillicfontsf{Arial}[Script = Cyrillic] % Шрифт без засечек для кириллицы
    % \usepackage{unicode-math} %\setmathfont
